\documentclass{article}
\usepackage[margin=0.5in]{geometry}
\usepackage{color}
\usepackage{setspace}
\usepackage{graphicx}
\setlength{\intextsep}{0.5cm}
\usepackage{textcomp}
\usepackage{setspace}
\usepackage{palatino}
\usepackage{hyperref}
\usepackage{fancyvrb}

\hypersetup{
	colorlinks=true, %set true if you want colored links
	linktoc=all,     %set to all if you want both sections and subsections linked
	linkcolor=blue,  %choose some color if you want links to stand out
	urlcolor=blue
}


\begin{document}

\title{Introduction to Biocomputing \\ Installation Instructions for Windows/PC}
\author{Stephanie Spielman \\ \footnotesize{Email: stephanie.spielman@gmail.com}}
\date{}
\maketitle{}

Unfortunately, PCs are not set up well for programming and scientific computing. Therefore, you'll need to install a virtual Linux machine on your computer. This process involves downloading some very large files (1-3 GB), so \textbf{set aside several hours to download and install everything!}. These instructions have been adapted from \emph{Practical Computing for Biologists} by Steven Haddock and Casey Dunn.

If you need help as you go along, check out this useful online guide: \\ \href{http://www.psychocats.net/ubuntu/virtualbox}{http://www.psychocats.net/ubuntu/virtualbox} (with screenshots of the process!).


\begin{enumerate}
	\item Install VirtualBox, a software that lets you run virtual machines with different operating systems on your computer. Download VirtualBox from here: \href{https://www.virtualbox.org/wiki/Downloads}{https://www.virtualbox.org/wiki/Downloads}. Download the appropriate binary file for your operating system from this page, and install using the default installer settings according to the instructions.
	
	\item Install a Linux operating system (OS). We recommend that you install Ubuntu (select Ubuntu Desktop), available here: \\ \href{http://www.ubuntu.com/download}{http://www.ubuntu.com/download}. Download the Ubuntu disk image from this website. You'll have two options, 32-bit or 64-bit, so download the one that matches your host machine (you can always use Google to find out what your machine is, but likely you'll want the 64-bit). Note that a 64-bit OS will only run on a 64-bit machine, but a 32-bit OS will run on either. \textbf{This download may take 1+ hours, depending on your internet connection.}
	\\ Once Ubuntu has downloaded, don't do anything with it! Leave it in your Downloads folder (or wherever it downloaded to) for now.
	
	\item Open your downloaded VirtualBox and click the "New" button to setup your virtual machine. Follow the setup wizard instructions:
	\begin{itemize}
		\item Select a name for your Ubuntu system ("Ubuntu" works!)
		\item Select Linux as the operating system
		\item Select Ubuntu as the version
		\item Select \emph{at least} 512 MB base memory (RAM). You can certainly select more than this, but you should not exceed $1/4$ of the RAM on your computer. For example, if you have 4 GB of RAM, then you should select 1 GB for the virtual machine.
		\item In the Virtual Hard Disk window, click “New” to make a new virtual hard disk.
		\item Select VDI
		\item You now have to decide whether to let VirtualBox expand your virtual hard disk dynamically (as you fill it up) or whether to set a fixed limit on how much hard drive space VirtualBox can take up. We recommend a fixed limit at roughly 10-20 GB.
	\end{itemize}
	
	\item Install Ubuntu on the VirtualBox. Start your virtual machine in VirtualBox by double clicking the icon on the left panel, which will launch the First Run Wizard. From this window, select CD/DVD-ROM Device for Media type, click the folder icon under Media Source, click the Add button at the top of the window that pops up, and then browse to where you saved the Ubuntu file that you downloaded and select that. At this point, your virtual machine should start and begin Ubuntu install process. Ubuntu will guide you through this, so follow the instructions.
	
	\item Finally, you can (although optional) download a text editor.
		
	
\end{enumerate}

Remember - the Ubuntu virtual machine you installed is fully distinct from your regular operating system. This means that, by default, these machines won't interact, e.g.\ can't copy/paste between them. The Ubuntu virtual machine will be in its own window, like an application on your computer. If you want to have these machines interact, then you'll need to setup a Shared Folder via Guest Additions. Instructions for this step are available here: \\ \href{http://practicalcomputing.org/ubuntu}{http://practicalcomputing.org/ubuntu}. Note that you might also encounter some weird mouse behavior and screen size issues - this is normal, and installing Guest Additions should help with this problem.





\end{document}
